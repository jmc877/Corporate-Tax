
\documentclass[12pt]{article}
\usepackage{geometry} % see geometry.pdf on how to lay out the page. There's lots.
\geometry{a4paper} % or letter or a5paper or ... etc
% \geometry{landscape} % rotated page geometry

% See the ``Article customise'' template for come common customisations
\usepackage[parfill]{parskip} 
\usepackage{graphicx}
\usepackage{amssymb}
\usepackage{epstopdf}
\usepackage{fancyhdr}
\usepackage{enumitem}
\pagestyle{fancy}
\renewcommand{\headrulewidth}{0pt}
\lhead{\textbf{Corporate Tax}}
\rhead{\textbf{Problem Set 3}}
\lfoot{\footnotesize CT$\_$PS$\_$3$\_$2021S}
\DeclareGraphicsRule{.tif}{png}{.png}{`convert #1 `dirname #1`/`basename #1 .tif`.png}

%%% BEGIN DOCUMENT
\begin{document}


	\textbf{Code and Regs}: 
		\begin{itemize}
				\item
					\S\S 83(a) and (h); 351(a), (b), (d), and (g); 358(a), (d) and (h); 362(a) and (e)(2); 368(c); 1001(a) and (c); 1012; 1032; 1223(1) and (2) 
				\item
					1.351-1(a)(1)(i), (ii), and (a)(2), Ex. 3 \\
		\end{itemize}

Assume that all stock is common stock unless otherwise noted, and that the installment sales rules don't apply.
\\
\\
\textbf{Transfers of Property to Controlled Corporations}
\begin{enumerate}


	\item
		Individuals A, B, \& C organize a corporation (C Corp) to which A transfers inventory worth \$10,000 with a basis of \$5,000 in exchange for 100 shares of  stock. B transfers depreciable business equipment purchased 2 years ago worth \$10,000 with a basis of \$5,000 in exchange for 100 shares of  stock, and C transfers \$10,000 cash in exchange for 100 shares of  stock.  
			\begin{enumerate}
				\item How much gain or loss does A, B, or C realize and recognize?
				\item How much gain or loss does C Corp realize and recognize? What if C Corp uses its shares that it had repurchased for \$1/share?
				\item What's A, B, and C's basis and holding period in the shares received?
				\item What's C Corp's basis and holding period in the property received?
			\end{enumerate}


		\item Same as 1, except C did not transfer cash to the corporation. Instead, he received 100 shares of stock in exchange for \$10,000 worth of services to be performed for the corporation.
				\begin{enumerate}
							\item How much gain or loss does A, B, or C realize and recognize?
							\item How much gain or loss does C Corp realize and recognize? 
							\item What's A, B, and C's basis and holding period in the shares received?
							\item What's C Corp's basis and holding period in the property received?
						\end{enumerate}
						
						
	\item Same as 1, except C receives 100 shares of stock in exchange for \$5,000 cash and \$5,000 worth of services to be performed for the corporation. Reg. \S1.351-1(a)(2), Ex. 3.


	\item
	Same as 1, except C receives 100 shares of stock in exchange for \$10 cash and \$9,990 worth of services to be performed for the corporation.   \textit{See} Rev. Proc. 77-37, \S3.07 and \emph{Kamborian's Estate v. CIR}.
	
	
	\item
	Same as 1, except C had a contract with D that as soon as the corporation was formed and he received 100 shares of stock, C would sell the 100 shares to D for \$10,000.  What issue does this raise?
%Maybe not cash but HB and lowB property instead
	\item Same as 1, except that B's equipment has a basis of \$15,000 instead of \$5,000. 
	
	\item 
		Same as 1, except that A transfers land held 5 years for investment with a basis of \$4,000 and a FMV of \$8,000 and \$2,000 cash for 100 shares of stock.   What are two ways to allocate the basis and holding period to the shares received?  \emph{See} Rev. Rul. 68-55 and Rev. Rul. 85-164.


	\end{enumerate}
%\item HOW TO INTENTIALLY FAIL 351; DEF OF 368C AND REV RUL; NON-VOTING HELD BY NON TRANSFEROR



%	Same as 1, except C decides immediately after receiving the 100 shares that he will give them to D, and he does so.
%	\item
%	Same as 1, except C, knowing that he wants to make a gift of the 100 shares to D, instructs the corporation to issue the 100 shares directly to D.
%
%	\item
%	Same as 1, except the corporation already was in existence prior to the transfers by A, B, \& C, and D owns throughout 100 shares of the corporation.
%	\item
%	Same as 1, except the corporation already was in existence prior to the transfers by A, B, \& C, and A prior to the transfers owns 500 shares of the stock.
%	\item Same as 1, except A's inventory is worth \$15,000 and C transfers \$5,000 cash. C
%	had rendered \$5,000 worth of services to A, the services having no relation to the assets transferred or to the business of the corporation.
%	\item
%	Same as 1, except B's equipment is subject to a \$3,000 mortgage. B receives from the corporation in exchange for his equipment 70 shares of stock worth \$7,000, and the corporation assumes the mortgage.
%	\item
%	What is the basis of the stock received by A, B, \& C from the corporation in numbers 1, 2, 11, 12, 13, and 14, above?


 \textbf{Boot}
	\begin{enumerate}[resume]
		\item 
			Same as 1, except B receives from the corporation in exchange for his equipment 70 shares of stock worth \$7,000, and 5-year term notes worth \$3,000.
	
		\item
			Same as 1, except B receives from the corporation in exchange for his equipment 30 shares of stock worth \$3,000 and 5-year term notes worth \$7,000.	
		
		\item  
				Same as 1, except B transfers two pieces of equipment, E1 and E2, in exchange for 70 shares worth \$7,000 and \$3,000 cash.  Both E1 and E2 have a basis of \$5,000, and E1 has a FMV of \$7,000 and E2 has a FMV of \$3,000.  \emph{See} Rev. Rul. 68-55.
				
		\item 
			Same as 1, except that B receives 20 shares of C Corp preferred stock worth \$2,000 whose dividend varies by the price of a bushel of corn and 80 shares of common stock.
			
		\item
			Describe the tax gambit addressed in GLAM  2020-005.
	\end{enumerate}
	


	\textbf{Liabilities}
	
	\begin{enumerate}[resume]
		
		\item Same as 1, except B's equipment is subject to a \$3,000 mortgage. B receives from the corporation in exchange for his equipment 70 shares of stock worth \$7,000, and the corporation assumes the mortgage.
		
		\item 	Same as 1, except B's equipment is subject to a \$7,000 mortgage. B receives from the corporation in exchange for his equipment 30 shares of stock worth \$3,000, and the corporation assumes the mortgage.

		\item Same as 1, except B's equipment is subject to a \$3,000 mortgage and C Corp also assumes \$4,000 of B's accounts payable that were incurred in B's ordinary course of business.  B is a cash method taxpayer.  B receives from the corporation in exchange for his equipment 30 shares of stock worth \$3,000, and the corporation assumes the mortgage and accounts payable.


	\end{enumerate}
	
	






	\end{document}	  