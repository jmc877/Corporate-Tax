
\documentclass[12pt]{article}
\usepackage{geometry} % see geometry.pdf on how to lay out the page. There's lots.
\geometry{a4paper} % or letter or a5paper or ... etc
% \geometry{landscape} % rotated page geometry

% See the ``Article customise'' template for come common customisations
\usepackage[parfill]{parskip} 
\usepackage{graphicx}
\usepackage{amssymb}
\usepackage{epstopdf}
\usepackage{fancyhdr}
\usepackage{enumitem}
\pagestyle{fancy}
\renewcommand{\headrulewidth}{0pt}
\lhead{\textbf{Corporate Tax}}
\rhead{\textbf{Problem Set 5: Redemptions}}
\lfoot{\footnotesize CT$\_$PS$\_$5$\_$2021S}
\DeclareGraphicsRule{.tif}{png}{.png}{`convert #1 `dirname #1`/`basename #1 .tif`.png}

%%% BEGIN DOCUMENT
\begin{document}


	\textbf{Code and Regs}: 
		\begin{itemize}
				\item
					\S\S 302(a), (b)(1)-(4), (b)(6), (c), (d); 312(n)(7); 317(b); 318 (skim references to trusts); 1059(e)(1)(A) and (e)(2)
				\item
					\S\S  1.302-2(a)-(c); 1.302-3; 1.302-4 (skim only)
					\\
		\end{itemize}

	

\begin{enumerate}


	\item 
		C Corp is owned by the following individuals (and shares owned): A (100); B (200); C (300); D (400), and E (300).  A and B are C's children, D is C's wife, and E is C's father.  How many shares does each own directly and constructively under \S 318? 
	
	\item 
		C Corp is owned 50\%-50\% by individuals A and B.  C Corp owns 70\% of D Corp.  B owns 100\% of E Corp.  I suggest drawing an ownership diagram.
		\begin{enumerate}
			\item How much does A own of D Corp and E Corp?
			\item How much of E Corp does D Corp own?
			\item How much of C Corp does E Corp own? 
			\item How much of D Corp does E Corp own?
		\end{enumerate}

	\item 
		C Corp has 1,000 common voting shares outstanding, and A owns 500 shares.  C Corp redeems 200 of A's shares.  Does the redemption qualify as substantially disproportionate under \S302(b)(2)?   
		
	\item
		Briefly explain the function of \S302(b)(2)(D).
	
	\item	
	
	V Corp is owned entirely by unrelated individuals J, K, and L, and has three classes of stock outstanding: voting common stock (CS), a class of voting preferred stock (VPS), and a class of nonvoting preferred stock (NVP). Each share of the VPS has the same voting rights as one share of the CS. J owns 200 shares each of the CS and the VPS. K owns 100 shares of each of the three classes of stock. L owns 100 shares of the NVP. Assume that V Corp has sufficient amount of E\&Ps.
		
		\begin{enumerate}
		\item  V Corp redeems 100 shares each of J's CS and VPS.  \emph{See} Rev. Rul. 75-502.
		\item V Corp redeems 20 shares of K's CS, 80 shares of VPS, and 50 shares of NVP.
		\item V Corp redeems 50 shares of K's CS, VPS, and NVP.
		\end{enumerate}
		
	
	\item
		A owns 800 shares and B owns the remaining 200 shares of the common stock of C Corp.  A is B's father.  A's basis in his shares is  \$80,000, and the FMV of the C Corp stock is \$1MM.  As part of succession planning, C Corp redeems all of A's shares for \$800,000. 
		% % Here add gift and then redemption
		
			\begin{enumerate}
			\item Does this redemption qualify under \S302(b)(3)?  Why not? What happens to A's basis in the redeemed shares?  \emph{See} \S1.302-2, Ex. 2.
			\item What can A do to bring the redemption under  \S302(b)(3).  What are the consequences to A and B?
			\item Same as (a) but A receives a note in exchange for his shares with the interest rate pegged to the 10-year treasury rate plus 3\%.
			\end{enumerate}
		
	\item 
	
	What would be the result if in Rev. Rul. 81-289, the corporation redeemed 1.5\% of its shares (15,000x/1,000,000x)? Are there any enforcement problems with redemptions by public corporations?
	
	\item When a shareholder invests in a mutual fund or requests redemption from a fund, the purchase or redemption price of each share is its net asset value (NAV), which is the assets under management (AUM) divided by the numbers of shares outstanding.  For example, if a fund has an AUM of \$1,000, and 100 shares outstanding, its NAV would be \$10/share (\$1,000 AUM / 100 shares).  
	
	Assume that mutual fund has 10 unrelated shareholders (A through J) each owning 10 shares (a total of 100 shares outstanding), and each share is worth \$10.  Assume that each shareholder has a basis of \$1 in each share.  What are the tax consequences to A in the following alternatives? 
		\begin{enumerate}
		\item A redeems all of its shares for \$100.
		\item A redeems one of its shares for \$10.
		\item A redeems one of its shares for \$10, and simultaneously B and C redeem all of their shares for \$100 each.  Please answer the question before looking at \S302(b)(5).
		\end{enumerate}	
	\item	
		What are the tax goals of the seller and purchaser in \emph{Zenz} and similar bootstrap acquisitions?
		
	\item Describe the transaction undertaken in \emph{H.J. Heinz Co. v. US}.  
		\begin{enumerate}
		\item Read \S1059(e)(1)(A)(ii) and and Reg. \S1.1059-1(e)(a).   How would those provisions affect the transaction today?
		\end{enumerate}
		
	\item	
		What is the proposal regarding the taxation of buybacks discussed in the Hemel article on the class web page?
	\end{enumerate}
	
	






	\end{document}	  