
\documentclass[12pt]{article}
\usepackage{geometry} % see geometry.pdf on how to lay out the page. There's lots.
\geometry{a4paper} % or letter or a5paper or ... etc
% \geometry{landscape} % rotated page geometry

% See the ``Article customise'' template for come common customisations
\usepackage[parfill]{parskip} 
\usepackage{graphicx}
\usepackage{amssymb}
\usepackage{epstopdf}
\usepackage{fancyhdr}
\usepackage{enumitem}
\pagestyle{fancy}
\renewcommand{\headrulewidth}{0pt}
\lhead{\textbf{Corporate Tax}}
\rhead{\textbf{Problem Set 7: Liquidations}}
\lfoot{\footnotesize CT$\_$PS$\_$7$\_$2021S}
\DeclareGraphicsRule{.tif}{png}{.png}{`convert #1 `dirname #1`/`basename #1 .tif`.png}

%%% BEGIN DOCUMENT
\begin{document}


	\textbf{Code and Regs}: 
		\begin{itemize}
				\item
					\S\S 168(k) (skim); 267(a)(1); 331; 332; 334; 336; 337; 346; 381(a)(1) (skim); and 1504(a)(2) (skim).	Note, to answer some of the problems, you may have to consult the unabridged versions of sections 336 and 337.
					
				\item
					\S\S  1.331-1(e); 1.332-2(b)
					\\
		\end{itemize}

	

\begin{enumerate}

		
	
	\item 
		Individual A owns 100\% of the 1,000 shares of C Corp.  A acquired the C Corp shares in two separate transactions and has a basis of \$20,000 in one block of 500 shares and a basis of \$100,000  in the other block of 500 shares.   What are the consequences to C Corp and A in the following scenarios?    
		\begin{enumerate}
			\item C Corp liquidates and distributes \$150,000 to A. \textit{See} Rev. Rul. 85-48.
			\item C Corp liquidates and pursuant to a plan distributes \$75,000 to A this year and \$75,000 to A in the following year.  \textit{See} Rev. Rul. 85-48.
			\item  What happens to C Corp's E\&Ps? \S381(a)(1).
		\end{enumerate}

	\item 
		Individual A owns 100\% of the 1,000 shares of C Corp.  The shares were acquired at the same time and have a basis of \$100,000.  What are the consequences to C Corp and A in the following scenarios?   Assume that \S336(d) doesn't apply unless otherwise stated.   
		
		\begin{enumerate}	
			\item C Corp liquidates and distributes property to A with a basis of \$100,00 and a FMV of \$150,000.  
			\item C Corp liquidates and distributes property to A with a basis of \$100,00 and a FMV of \$150,000 subject to a liability of \$75,000.
			\item C Corp liquidates and distributes property to A with a basis of \$100,00 and a FMV of \$150,000 subject to a liability of \$125,000.
			\item C Corp liquidates and distributes property to A with a basis of \$100,00 and a FMV of \$150,000 subject to a liability of \$175,000.  Rev. Rul. 2003-125.
			% % What's basis of property? Probbly  FMV; see B&E
			\item  C Corp liquidates and distributes property to A with a basis of \$150,00 and a FMV of \$75,000. 
		\end{enumerate}
		
	\item	
		A and B own 80\% and 20\%, respectively, of the 1,000 shares of C Corp.  A has a basis of \$80,000 in the C Corp shares, and B a basis of \$20,000.  C Corp owns two assets, Thing \#1 and Thing \#2.  Thing \#1 has an AB of \$10,000 and a FMV of \$20,000, and Thing \#2 has an AB of \$100,000 and a FMV of \$80,000. Both assets were purchased for cash a decade ago.  What are the consequences to C Corp, A, and B in the following scenarios if C Corp adopts a plan of liquidation?    
	
		\begin{enumerate}	
				\item A and B are individuals, and C Corp sells its assets to a 3rd party for their FMV and distributes the proceeds in liquidation.
				\item A and B are individuals, and C Corp distributes its assets pro rata in liquidation, \emph{i.e.}, A receives an 80\% interest in each asset.
				\item A and B are individuals, and C Corp distributes Thing \#1 to B and Thing \#2 to A in liquidation.
				\item A and B are individuals, and C Corp distributes its assets pro rata in liquidation, but Thing \#1 was acquired three years ago in a 351 transaction.  Review \S362(e)(2).
				\item A and B are individuals, and C Corp distributes Thing \#1 to B and Thing \#2 to A in liquidation, except that Thing \#1 has an AB of \$10,000 and a FMV of \$20,000, and Thing \#2 has an AB of \$90,000 and a FMV of \$80,000.
		
			
		\end{enumerate}
	
	\item	
			A Corp, a C corporation, owns 100\% of the 1,000 shares of C Corp.  A has a basis of \$80,000 in the C Corp shares.  C Corp owns two assets, Thing \#1 and Thing \#2.  Thing \#1 has an AB of \$50,000 and a FMV of \$20,000, and Thing \#2 has an AB of \$50,000 and a FMV of \$80,000. What are the consequences to C Corp and A in the following scenarios if C Corp adopts a plan of liquidation?   
	
			\begin{enumerate}	
						\item C Corp distributes its assets to A Corp in liquidation.  
						\item Same as previous question. What happens to C Corp's E\&Ps?
						\item A Corp is Fordham University, a tax-exempt entity, and C Corp distributes its assets to A in liquidation.
						\item C Corp owes A Corp \$20,000, and prior to completing the liquidation, C Corp distributes a one-fifth interest in Thing \#2 in satisfaction of the debt.

					
				\end{enumerate}
	
	\item What is the policy rationale of \S336(d)(2).  Is it still necessary given the rules in \S362(e)(2)?  Comment.
	
	\end{enumerate}
	
	






	\end{document}	  